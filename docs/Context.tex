\documentclass[english]{article}
\usepackage[utf8x]{inputenc}
\usepackage[T1]{fontenc}
\usepackage{babel}
\usepackage{amsmath}
\usepackage{graphicx}
\usepackage{hyperref}
\usepackage{titling}
\usepackage{fancyhdr}

\setlength{\droptitle}{-2em} 
\posttitle{\vspace{-3em}} 

\pagestyle{fancy}
\fancyhf{}
\renewcommand{\headrulewidth}{0pt}
\setlength{\headheight}{40pt} 
\usepackage[a4paper,top=2cm,bottom=2cm,left=3cm,right=3cm,marginparwidth=1.75cm]{geometry}

\begin{document}

%-------------------------------------------------------%

\title{\bf How Data and Analytics Influence Student Investment Choices}
\date{}
\maketitle
\thispagestyle{fancy}

%-------------------------------------------------------%
\section{Introduction}
Over the past few years, digital platforms like Vanguard, eToro or Trading212 have allowed people to invest
with just a few taps.This has encouraged many students to start investing earlier, 
often relying on app-based data such as portfolio performance or stock market analytics to advise their decisions.
However, whilst these tools provide access to this data, they also assume that the user is able to interpret 
the financial information and make fact based investment decisions.

Experts agree that financial knowledge is strongly related to better financial behaviours
(Rodriguez-Correa, 2025). However, research has shown that many young adults struggle
with financial literacy, which can lead to poor financial decision making and have potential lifelong 
impacts (Lusardi, 2019). Financial literacy refers to the ability of indivudlas to make clear decisions 
about their financial life based on judgements formed from training in the topics about money, credit, 
savings and many more (Rodriguez-Correa, 2025).

However, it is not just financial literacy that is important. With the rise of these investment platforms relying on
visualisations to present the data in a clear and concise manner, digital financial literacy (DFL) has become increasingly important.
Studies show that a higher digital literacy is linked to greater financial confidence and more informed decisions. 
For instance, Basar et al. (2025) found that digital financial literacy, alongside the use of FinTech platforms, 
has a positive relationship with saving habits and responsible financial behaviour. 
While existing research highlights the important of financial literacy and DFL, few studies have explored how these
infuence actual investment behaviour in student populations.

This study will explore how University of Exeter students use investment platforms and how their financial
and digital literacy affect their investment behaviour. By understanding how certain students engage with data 
and analytics, the research aims to discover whether these digital tools are effectively guiding better decisions or
introducing other challenges. The findings from this research could help improve university initiatives on financial
education and inform how investment platforms present data to young investors.

%-------------------------------------------------------%
\section{Data Visualisation Research}

%-------------------------------------------------------%
% Figures and tables
\clearpage

%-------------------------------------------------------%
\clearpage
% Include your references here.
\bibliographystyle{plain}
\bibliography{references}

%-------------------------------------------------------%

%-------------------------------------------------------%
\clearpage
% Include your appendix here, if necessary.
% \section*{Appendix}
% % Appendix (optional, unlimited)
% % Code snippets, supplementary figures, tables, or additional materials.
% % Will not be marked but can provide supporting evidence.

%-------------------------------------------------------%
\end{document}
