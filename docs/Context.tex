\documentclass[english]{article}
\usepackage[utf8]{inputenc}
\usepackage[T1]{fontenc}
\usepackage{babel}
\usepackage{amsmath}
\usepackage{graphicx}
\usepackage{hyperref}
\usepackage{titling}
\usepackage{fancyhdr}
\usepackage{float}
\usepackage{cite}  
\usepackage{setspace}
\usepackage[labelfont=bf,textfont=bf]{caption}


\pagestyle{fancy}
\fancyhf{}

\renewcommand{\headrulewidth}{0pt}
\setlength{\headheight}{40pt} 
\usepackage[a4paper,top=2cm,bottom=2cm,left=2cm,right=2cm,marginparwidth=1.75cm]{geometry}

\begin{document}

%-------------------------------------------------------%

\title{\bf Navigating FinTech: The Role of Financial and Digital Literacy in Student Investment Decisions}
\date{}
\maketitle
\thispagestyle{fancy}

%-------------------------------------------------------%
\vspace{-8em}
\section{Introduction}

\vspace{-0.5em}

Over the past few years, digital investment platforms like Vanguard, eToro or Trading212 have made it easier for people to invest\cite{Spear2024-iv}. 
This has encouraged many students to start investing earlier, 
usually relying on app-based data such as portfolio performance or stock market analytics to guide their decisions\cite{Freibauer2024-qo}.
However, while these tools provide access to this data, they assume that the user is able to interpret 
the financial information and make informed investment decisions.

However, many young adults struggle with financial literacy, which is the ability of individuals to make clear decisions 
about their financial life based on knowledge from education, in areas surrounding money, credit, 
savings and many others\cite{Rodriguez-Correa2025-ld}.
Research has found that increased financial knowledge can be strongly related to better financial behaviours\cite{Rodriguez-Correa2025-ld}.
This lack of financial literacy, within student populations is so significant the UK government recently introduced mandatory financial 
education into school curriculum up until 18 years old\cite{UK_Government}. The lack of education can lead to poor financial decision making, introducing potential lifelong 
impacts and financial pressures affecting quality of life\cite{Lusardi2019-pt}.

However, it is not just financial literacy that carries importance. With investment platforms relying on
visualisations to present the data in a clear and concise manner, digital literacy has become pivotal.
Basar et al.\cite{Basar2025-xg} found that digital financial literacy, alongside the use of FinTech platforms, 
has a positive impact on saving habits and responsible financial behaviour. 
This existing research highlights the importance of financial and digital literacy, however, few studies have 
discussed their influence on investment behaviours in student populations.

This study will investigates how University of Exeter students interact with investment platforms, along with how their financial
and digital literacy affect investment behaviours. By understanding how students engage with these platforms,
the research aims to discover whether digital tools are effectively influencing better decisions or
introducing other challenges. The findings from this research could help improve financial
education initiatives and inform how investment platforms present data to investors.

\vspace{-1em}

%-------------------------------------------------------%
\section{Data Visualisation Research}

\vspace{-0.5em}

Figures \ref{Fig1}, \ref{Fig2}, and \ref{Fig3} show visualisations from existing research
that evaluate financial literacy, digital literacy, and investment behaviours. These figures influenced
the design of visualisations within this research project, ensuring clarity and effective communication of findings.

\begin{figure}[H]
    \centering
    \includegraphics[width=0.8\textwidth]{../figures/ContextFig3.png} 
    \caption{Box Plot by Luksander et al.\cite{Luksander2014-in}}
    \label{Fig1}
\end{figure}

\clearpage
\begin{figure}[H]
    \centering
    \includegraphics[width=0.6\textwidth]{../figures/ContextFig1.png} 
    \vspace{-1em}
    \caption{Bar Chart by Lusardi et al.\cite{Lusardi2019-pt}}
    \label{Fig2}
\end{figure}

\vspace{-2em}

\begin{figure}[H]
    \centering
    \includegraphics[width=0.55\textwidth]{../figures/ContextFig2.png} 
    \vspace{-1em}
    \caption{Correlation Matrix with Histograms by Cannistra et al.\cite{Cannistra2022-lr}}
    \label{Fig3}
\end{figure}

\vspace{-2.5em}

%-------------------------------------------------------%

% References
\begin{spacing}{0.2}
    \bibliographystyle{IEEEtran}
    \bibliography{../references/context_references}
\end{spacing}

%-------------------------------------------------------%
\end{document}
