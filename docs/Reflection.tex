\documentclass[english]{article}
\usepackage[utf8]{inputenc}
\usepackage[T1]{fontenc}
\usepackage{babel}
\usepackage{amsmath}
\usepackage{graphicx}
\usepackage{hyperref}
\usepackage{titling}
\usepackage{fancyhdr}
\usepackage{float}
\usepackage[round,comma]{natbib}

\pagestyle{fancy}
\fancyhf{}

\renewcommand{\headrulewidth}{0pt}
\setlength{\headheight}{36pt} 
\usepackage[a4paper,top=2cm,bottom=2cm,left=2cm,right=2cm,marginparwidth=1.75cm]{geometry}

\begin{document}

%-------------------------------------------------------%
\title{\bf Peer-Review and AI-Use Reflection}
\date{}
\maketitle
\thispagestyle{fancy}

\vspace{-8em}

\section{Use of AI in Research}

\vspace{-0.5em}

Throughout this research, I used three AI tools: Gemini for initial topic brainstorming, plan development, 
and code reviews, Perplexity for summerisation of large policy documents and for reviewing
and providing feedback on drafts and finally Github Copilot for real-time coding bug-fixes.

\vspace{-1em}

\subsection{Idea Generation, Planning and Review}

\vspace{-0.5em}

Gemini proved to be valuable in the brainstorming phase; identifying 10 topic
areas for potential research, however, these were broad which encouraged independent thinking to narrow the ideas
down to my final research question 'Navigating FinTech: The Role of Financial and 
Digital Literacy in Student Investment Decisions'. AI provided a good starting place however, to verify
the topic discussed I used Google Scholar and the University library to find existing literature to help 
identify gaps that my research could fill.

To understand University requirements for ethical research, I chose to use Perplexity to summarise and review 
ethics and policy documentation. This helped the efficiency of my research by extracting key information from these large 
documents effectively. I reviewed the source texts to ensure accuracy and find any missing information or
hallucinations in AI responses. Perplexity was also very useful when providing feedback on structure, 
clarity and grammar within my drafts. I found that AI provided clear suggestions for improving my writing, however, 
I had to be careful to be critical of its recommendations as it could change the intended meaning 
of my report.

\vspace{-1em}

\subsection{Coding Tasks}

\vspace{-0.5em}

Gemini was also used when performing code reviews to help identify any issues with data representation, 
or clean and optimise my code. To verify the recommendations, 
I implemented everything line by line, rather than generating the code and copying it verbatum, 
ensuring I understood any modifications being made.

Furthermore, whilst writing my code, Github Copilot was enabled allowing for 
real-time bug-fixes; however, I found the constant code suggestions distracting as it would often recommend
complex code where simpler approaches would be a better alternative. This highlighted an important limitation;
over-reliance on autocomplete which can reduce understanding of how and why the code works, potentially
limiting learning.

\vspace{-1.25em}


\section{Peer Review}

\vspace{-0.75em}

\subsection{Feedback}

\vspace{-0.5em}


The peer review process provided valuable feedback to improve my research. 
The feedback highlighted areas where I could improve clarity, especially in my figures and explanations of 
statistical analysis. For example, I was advised to add more detailed captions to my figures to help readers
interpret the data more easily. Furthermore, I received suggestions to increase the size of some of my figures
to ensure they were easily readable. As well as, some minor grammatical feedback to enhance the overall
quality of my writing alongside including a clearer explanation of my research focus within the introduction.

\vspace{-1em}

\subsection{Actions}

\vspace{-0.5em}

I have made the recommended changes to my figures, including adding more detailed captions and increasing their size.
Additionally, I have reviewed the explanation of some of my statistical analysis to ensure it is clear and accessible
to readers. Finally, I have proofread addressing any grammatical issues.
As well as, revising my introduction to clearly state my research focus and objectives.

\vspace{-1.25em}

\section{Focus for Review}

\vspace{-0.5em}

I believe that my research could benefit from further review in some key areas; the first being
whether the questions that I used to calculate financial and digital literacy, in addition to investment
behaviour, were the most accurate to quantify such characteristics. As I am still uncertain if there are 
other questions that could represent these skills more accurately. For example, lots of my questions
also test mathematical ability, which may not be a true reflection of someone's financial or digital literacy.

Also, my research would benefit from reviewing whether my references to University of Exeter policies are 
accurate. While I carefully reviewed these documents with the help of Perplexity, I found it challenging to 
understand where and when to apply particular policy clauses within my research design. Expert guidance
on the correct institutional interpretation would strengthen my Data Management and Ethics sections.

Futhermore, I would appreciate feedback on my statistical analysis methods, specifically whether the tests
I chose were the most appropriate for my data and research questions. As well as, whether including the correlation plot
rather than the model diagnostics plots within my final report was best to illustrate my findings.

\end{document}