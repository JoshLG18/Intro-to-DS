\documentclass[english]{article}
\usepackage[utf8]{inputenc}
\usepackage[T1]{fontenc}
\usepackage{babel}
\usepackage{amsmath}
\usepackage{graphicx}
\usepackage{hyperref}
\usepackage{titling}
\usepackage{fancyhdr}
\usepackage{float}
\usepackage{setspace}
\usepackage{cite} 

\pagestyle{fancy}
\fancyhf{}

\renewcommand{\headrulewidth}{0pt}
\setlength{\headheight}{36pt} 
\usepackage[a4paper,top=2cm,bottom=2cm,left=2cm,right=2cm,marginparwidth=2cm]{geometry}

\begin{document}

%-------------------------------------------------------%
\title{\textbf{Data Management Plan, and Ethics Review}}
\date{}
\maketitle
\thispagestyle{fancy}



%-------------------------------------------------------%
\vspace{-8.75em}

\section{\textbf{Data Management Plan}}

\vspace{-0.5em}

\subsection{\bf Public Good}

This research supports the University of Exeter's aim for all of it's activities to have
positive impacts on its stakeholders, as well as improve the living and studying quality for students\cite{University_of_Exeter2021-td}.

Previous research shows that young adults struggle with financial literacy\cite{Lusardi2019-pt}, which has impacts on 
investment behaviours\cite{Rodriguez-Correa2025-ld}. Findings from this research will influence the development of 
financial education initiatives across all University courses, benefiting thousands of students. 
The potential reduction in financial pressures will meet the University's purpose
through public benefit research as outlined in the University of Exeter's Research Ethics Framework\cite{University_of_Exeter2022-ll}.

\vspace{-0.5em}

\subsection{\bf Data Description}
This study will collect primary quantitative survey data from students at the University of Exeter. 
The final anonymised dataset will include: demographic data, consisting of background information on University year, 
field of study and investment experience.
As well as, literacy assessments assessing objective and subjective knowledge based around finance and digital skills. 
Finally, behavioural data will capture how students use investment platforms and allows the quantification 
of investment behaviours. No identifiable information is collected, ensuring GDPR compliance\cite{European_Union2020-cw} as required
by the Research Ethics Framework\cite{University_of_Exeter2022-ll}.

This dataset will allow for multivariate analyses to explore relationships between financial literacy, 
digital literacy, and investment behaviours as well as comparisons across student subgroups.

\vspace{-0.5em}

\subsection{\bf Data Collection}

Data collection will occur through a survey on Microsoft Forms, as it has University approval, aimed at University of Exeter
students. To ensure the quality of data collection, 10 -- 20 students will complete a pilot study to refine questions and 
validate completion time, and collection methods. Recruitment will occur via the University email list, campus presence,
and students' guild channels, aiming for 1000 participants to ensure statistical power.

The dataset will follow FAIR principles\cite{Go_fair2017-sd}, with consistent naming conventions, 
and a CSV format with UTF-8 encoding. Anonymity will promote honest responses to improve the validity of the analyses. 
To ensure institutional consent compliance, the University Participant 
Information Sheets and Consent Forms will be used to gain informed consent from participants as required by the Research Ethics Framework\cite{University_of_Exeter2022-ll}.

\vspace{-0.5em}

\subsection{\bf Data Storage and Security}

The research team will securely store all data on the University-approved Research Data Storage\cite{University_of_Exeter2025-mq}, 
allowing for encryption, access control and automated backups. 
This ensures Code of Good Practice in the Conduct of Research compliance\cite{University_of_Exeter2024-ih}.

Only the lead researcher and their supervising team will have access and only through their official University credential before it is
stored openly.
The team will export the data directly from the survey platform to this secure storage, to mitigate potential data interception complications.

In the event of data transfer, only secure and encrypted channels will
be permitted, with no storage on personal or unencrypted devices.
These measures will ensure confidentiality, integrity, and GDPR compliance\cite{European_Union2020-cw}.
Futhermore, a data protection impact assessment 
will be completed as required by the Research Ethics Framework\cite{University_of_Exeter2022-ll} to identify and mitigate any potential risks.

Finally, primary data will be retained for 10 years post-completion as stated within the
Code of Good Practice in the Conduct of Research\cite{University_of_Exeter2024-ih}, allowing for further research or replication.

\vspace{-0.5em}

\subsection{\bf Data Management, Documentation, and Curation}

Documentation including variable names, how responses are encoded along with the response categories shall be stored
in a data dictionary alongside the data file.
Also, a file containing metadata including survey structure, collection dates, consent procedures, 
and data cleaning steps will be stored with the data to allow for reproduction.
File naming conventions and folder structures will also be consistent and descriptive, to ensure others can understand 
and use the data effectively. 

To support long-term access and transparency, 
the dataset and documentation will be prepared for archiving in University's Open Research Exeter (ORE) repository\cite{University_of_Exeter2017-wk},
to allow for open access and reuse by other researchers.

\vspace{-0.5em}

\subsection{\bf Data Preservation}
As stated in Section 1.5, final datasets and documentation will be archived in the University's ORE repository\cite{University_of_Exeter2017-wk}.
Data will be stored in multiple formats (CSV, TXT, XLSX) with metadata to support reusability.
Access conditions will be defined based on ethical approval and consent. Also, data shall be retained for a minimum of 10 years
as stated in Section 1.4.

\vspace{-0.5em}

\subsection{\bf Data Sharing and Access}

As discussed in Section 1.5, the ORE repository\cite{University_of_Exeter2017-wk} will store the data and the analysis 
to support research transparency as it is openly accessible. 
Reporting analysis findings to University wellbeing services, student support services, 
and the Students' Guild will help inform practical interventions. Finally, if participants request a summary of the results, they will
be provided with one to comply with GDPR\cite{European_Union2020-cw}.

%-------------------------------------------------------%
\vspace{-1em}

\section{\textbf{Ethics and Privacy Review}}

\subsection{Ethics and Privacy}

This study presents a low level of ethical risk. It involves voluntary participation in an anonymous 
online survey with no collection of sensitive data. All participants will 
receive an information sheet and consent form outlining the purpose of the study, their rights, 
and how the data will be used. To mitigate any other risks related to privacy,
no IP addresses will be collected, and all data will be stored securely on University-approved 
platforms with controlled access. Along with this, to avoid any potential distress or anxiety caused financial questions,
students will be provided with resources for financial support and counselling services, 
as well as the option to withdraw from the survey and study at any time without penalty.
Also, ethical approval will be acquired through the University of Exeter’s research 
ethics process to ensure full compliance with institutional standards.

\vspace{-0.75em}

\subsection{Relevant Policies}
This study complies with GDPR and the Data Protection Act 2018 governing lawful handling of 
personal data\cite{European_Union2020-cw},\cite{UK_Government2018-sd}.
It adheres to the University of Exeter's Research Ethics Framework\cite{University_of_Exeter2022-ll}, 
Ethics Policy\cite{University_of_Exeter2021-td}, 
Code of Good Practice in the Conduct of Research\cite{University_of_Exeter2024-ih}, and Open Access and 
Research Data Management Policy\cite{University_of_Exeter2017-wk}, which govern data storage, retention, and sharing.
All procedures follow these frameworks to ensure legal, ethical, and institutional compliance.

\vspace{-1em}
%-------------------------------------------------------%
% References
\begin{spacing}{0.5}  
    \bibliographystyle{IEEEtran}
    \bibliography{../references/plan_references}
\end{spacing}

%-------------------------------------------------------%
\end{document}