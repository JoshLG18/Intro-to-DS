\documentclass[english]{article}
\usepackage[utf8]{inputenc}
\usepackage[T1]{fontenc}
\usepackage{babel}
\usepackage{amsmath}
\usepackage{graphicx}
\usepackage{hyperref}
\usepackage{titling}
\usepackage{fancyhdr}

\pagestyle{fancy}
\fancyhf{}
\renewcommand{\headrulewidth}{0pt}
\setlength{\headheight}{40pt} 
\usepackage[a4paper,top=2cm,bottom=2cm,left=2cm,right=2cm,marginparwidth=1.75cm]{geometry}

\begin{document}

%-------------------------------------------------------%

\title{\bf Figures}
\date{}
\maketitle
\thispagestyle{fancy}

\vspace{-8em}

\begin{figure}[h!]
    \includegraphics[width=1\textwidth]{../figures/combined_platform_analysis.png} 
    \vspace{-2em}
    \textbf{\caption{Investment Platform Engagement Among University of Exeter Students (N = 1000).}}
  
    \textbf{Left:} This stacked bar chart shows a potential correlation between academic background and 
    investment activity. Business/Economics students demonstrated the highest 
    engagement, with $80\%$ reporting platform use, contrasting with other fields
    like Humanities/Arts, where most students reported no usage. This relationship could suggest that financial
    literacy is a key driver in initial investment participation.
    
    \vspace{0.5em}

    \textbf{Right:} Among all students, the equal largest 
    group ($380$) were non-investors, but for those who do invest, Trading apps 
    (e.g., Trading212, Vanguard) were the majority choice, used by approximately 380 
    students, nearly double the usage of Robo-advisors ($200$). This
    suggests a strong preference among the student population to have full control over
    their investments.
\end{figure}

%-------------------------------------------------------%

\begin{figure}[h!]
    \includegraphics[width=\textwidth]{../figures/score_histograms.png}
    \vspace{-2em}
    \textbf{\caption{Distribution of Financial and Digital Literacy Scores Among Students (N = 1000).}}

    This figure displays the distribution of scores on the financial and digital literacy assessments in the
    student population.

    \vspace{0.5em}

    \textbf{Financial Literacy (Left):} Financial literacy shows a normal distribution
    (bell-shaped), centered around a score of 3 to 4 out of 7. The highest frequency being 
    observed at a score of 3, with over 200 students achieving this score. This pattern 
    suggests that while most students have a moderate or foundational level of financial
    knowledge, a majority struggled to answer half the questions correctly,
    supporting existing research on low financial literacy among young adults.

    \vspace{0.5em}

    \textbf{Digital Literacy (Right):} This distribution is negatively skewed where, 
    the highest frequencies are concentrated 
    between scores of 4 and 6, with a large number of students scoring 7. 
    This indicates a generally higher level of digital literacy compared to financial
    literacy among the student population, which reinforces the need for financial education initiatives. 
   
\end{figure}

\clearpage

%-------------------------------------------------------%

\begin{figure}[h!]
    \includegraphics[width=\textwidth]{../figures/combined_box_score_analysis_with_anova.png}
    \vspace{-3em}
    \textbf{\caption{Financial Literacy in Student Subgroups with ANOVA and Tukey HSD (N = 1000).}}
    The analysis of financial literacy amongst student subgroups reveals that both 
    academic background and investment experience have clear impacts on the level financial 
    knowledge. 

    \vspace{0.25em}

    \textbf{Left:} The difference in financial literacy scores by study field was shown to be
    significant ($F = 300.31, p = 0.000$), with Business/Economics 
    students recording the highest median score $5/7$ and the lowest variance.
    Post-hoc Tukey HSD tests found that Business/Economics students had significantly higher
    financial literacy than all other fields (all $p < 0.05$), further reiterating the need for education initiatives across all courses.
    
    \vspace{0.25em}

    \textbf{Right:} Similarly, a significant difference was observed 
    across investment experience groups ($F = 7.62, p = 0.000$), where the non-investor group had 
    the lowest median score $3/7$. Post-hoc Tukey HSD tests reinforce that non-investors
    had significantly lower financial literacy than all other groups (all $p < 0.05$),
    suggesting that investment experience is associated with higher financial literacy.
\end{figure}

%-------------------------------------------------------%

\begin{figure}[h!]
    \includegraphics[width=\textwidth]{../figures/combined_correlation_and_feature_importance.png}
    \vspace{-2.5em}
    \textbf{\caption{Multivariate Correlation and Regression Analysis with Feature Importance (N = 1000).}}
    \textbf{Left:}
    The pearson correlation matrix shows clear positive linear relationships, 
    the highest being between financial literacy and being a Business/Economics student ($r=0.60$), 
    as well as with investment behaviour ($r=0.47$) and confidence in financial terms ($r=0.42$). 
    Digital literacy also shows some correlation with investment behaviour ($r=0.38$), indicating
    that higher literacy levels may be associated with better investment decisions.

    \vspace{0.25em}

    \textbf{Right:} The bar chart displays the variable importance from a linear
    regression model predicting investment behaviour ($R^2=0.41$), indicating that the model explains 41\%
    of the variation in investment behaviour. This suggests that
    other factors may play a significant role such as social or economic influences.
    The model shows that confidence in financial terms ($\beta = 0.62$) and reliance on data and analytics ($\beta = 0.58$) 
    are the strongest
    positive predictors. Interestingly, digital literacy ($\beta = 0.47$) shows a slightly greater impact than 
    financial literacy ($\beta = 0.40$), contradicting the correlations found from the above matrix, 
    suggesting that psychological factors and digital engagement are greater influencers of investment behaviour 
    than measured financial knowledge.
\end{figure}

\end{document}

